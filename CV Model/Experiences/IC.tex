\interprise{\href{https://pcs.usp.br/en/}{PCS-EPUSP \textcolor{maincol}{\faLink}} | \href{https://fapesp.br/en}{FAPESP \textcolor{maincol}{\faLink}}}{Suite a présentation au comité scientifique obtenu d'une subvention pour développer un projet de recherche}{Oct 2019 - Oct 2020}

\cvevent{}{}{Assistant de rechercher}{Recherche et Développement}
    {\begin{itemize}
      %  \item Travail intitule \textbf{A Proposal for a Concrete Syntax for Use Case}.
        \item Améliorer la représentation d'\textbf{Exigence Fonctionnelles (cas d'usage)} dans les processus \textbf{de Génie de Software}
        \vspace{-5pt}\item\textbf{l'Ingénierie Dirigée par les Modèles} (IDM) et implémenté le \textbf{\textit{métamodèle}} a l'aide de \textbf{Xtext}
        \vspace{-5pt}\item Projet \textit{open source} sur \href{}{GitHub \textcolor{maincol}{\faGithub}}
        %\item J'ai créé une syntaxe concrète et j'ai implémenté le \textbf{\textit{métamodèle}} a l'aide de \textbf{Xtext};
        %\item Il est \textbf{publié} au Symposium Brésilien sur les Systèmes d'Information (SBSI) \textcolor{maincol}{\href{https://sol.sbc.org.br/index.php/sbsi_estendido/article/view/15353}{\faFile}};
        %\item Le outil \textbf{UCWriter} est un projet \textit{open-source} sur GitHub \href{https://github.com/JoseColombini}{\textcolor{maincol}{\faGithub}}.
    \end{itemize}}{}{}{\begin{itemize}
        \item \href{https://sol.sbc.org.br/index.php/sbsi_estendido/article/view/15353}{Publiés \textbf{A Proposal for a Concrete Syntax for Use Case} \textcolor{maincol}{\faFile}}
    \end{itemize}}
    
    