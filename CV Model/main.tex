%


%-----------------------------------------------------------------------------------------------------------------------------------------------%
%	The MIT License (MIT)
%
%	Copyright (c) 2019 Jan Küster
%	Copyright (c) 2021 Jose Colombini

%
%	Permission is hereby granted, free of charge, to any person obtaining a copy
%	of this software and associated documentation files (the "Software"), to deal
%	in the Software without restriction, including without limitation the rights
%	to use, copy, modify, merge, publish, distribute, sublicense, and/or sell
%	copies of the Software, and to permit persons to whom the Software is
%	furnished to do so, subject to the following conditions:
%	
%	THE SOFTWARE IS PROVIDED "AS IS", WITHOUT WARRANTY OF ANY KIND, EXPRESS OR
%	IMPLIED, INCLUDING BUT NOT LIMITED TO THE WARRANTIES OF MERCHANTABILITY,
%	FITNESS FOR A PARTICULAR PURPOSE AND NONINFRINGEMENT. IN NO EVENT SHALL THE
%	AUTHORS OR COPYRIGHT HOLDERS BE LIABLE FOR ANY CLAIM, DAMAGES OR OTHER
%	LIABILITY, WHETHER IN AN ACTION OF CONTRACT, TORT OR OTHERWISE, ARISING FROM,
%	OUT OF OR IN CONNECTION WITH THE SOFTWARE OR THE USE OR OTHER DEALINGS IN
%	THE SOFTWARE.
%	
%
%-----------------------------------------------------------------------------------------------------------------------------------------------%


%============================================================================%
%
%	DOCUMENT DEFINITION
%
%============================================================================%

%we use article class because we want to fully customize the page and don't use a cv template
\documentclass[10pt,A4]{article}	


%----------------------------------------------------------------------------------------
%	ENCODING
%----------------------------------------------------------------------------------------

% we use utf8 since we want to build from any machine
\usepackage[utf8]{inputenc}		

%----------------------------------------------------------------------------------------
%	LOGIC
%----------------------------------------------------------------------------------------

% provides \isempty test
\usepackage{xstring, xifthen}

%----------------------------------------------------------------------------------------
%	FONT BASICS
%----------------------------------------------------------------------------------------

% some tex-live fonts - choose your own

%\usepackage[defaultsans]{droidsans}
%\usepackage[default]{comfortaa}
%\usepackage{cmbright}
\usepackage[default]{raleway}
%\usepackage{fetamont}
%\usepackage[default]{gillius}
%\usepackage[light,math]{iwona}
%\usepackage[thin]{roboto} 

% set font default
\renewcommand*\familydefault{\sfdefault} 	
\usepackage[T1]{fontenc}

% more font size definitions
\usepackage{moresize}

%----------------------------------------------------------------------------------------
%	FONT AWESOME ICONS
%---------------------------------------------------------------------------------------- 

% include the fontawesome icon set
\usepackage{fontawesome}
\usepackage[dvipsnames]{xcolor}
\usepackage{academicons}

% use to vertically center content
% credits to: http://tex.stackexchange.com/questions/7219/how-to-vertically-center-two-images-next-to-each-other
\newcommand{\vcenteredinclude}[1]{\begingroup
\setbox0=\hbox{\includegraphics{#1}}%
\parbox{\wd0}{\box0}\endgroup}

% use to vertically center content
% credits to: http://tex.stackexchange.com/questions/7219/how-to-vertically-center-two-images-next-to-each-other
\newcommand*{\vcenteredhbox}[1]{\begingroup
\setbox0=\hbox{#1}\parbox{\wd0}{\box0}\endgroup}


%TAGGGGGGGGGGG60
\newcommand{\cvtag}[1]{%
  \tikz[baseline]\node[anchor=base,draw=maincol,rounded corners,inner xsep=1ex,inner ysep=0.75ex,text height=1.5ex,text depth=.25ex, thick]{#1};
  \vspace{0.25ex}
}

% icon shortcut
\newcommand{\icon}[3] { 							
	\makebox(#2, #2){\textcolor{maincol}{\csname fa#1\endcsname}}
}	

% icon with text shortcut
\newcommand{\icontext}[4]{ 						
	\vcenteredhbox{\icon{#1}{#2}{#3}}  \hspace{2pt}  \parbox{0.9\mpwidth}{\textcolor{#4}{#3}}
}

% icon with website url
\newcommand{\iconhref}[5]{ 						
    \vcenteredhbox{\icon{#1}{#2}{#5}}  \hspace{2pt} \href{#4}{\textcolor{#5}{#3}}
}

% icon with email link
\newcommand{\iconemail}[5]{ 						
    \vcenteredhbox{\icon{#1}{#2}{#5}}  \hspace{2pt} \href{mailto:#4}{\textcolor{#5}{#3}}
}

%----------------------------------------------------------------------------------------
%	PAGE LAYOUT  DEFINITIONS
%----------------------------------------------------------------------------------------

% page outer frames (debug-only)
% \usepackage{showframe}		

% we use paracol to display breakable two columns
\usepackage{paracol}

% define page styles using geometry
\usepackage[a4paper]{geometry}

% remove all possible margins
\geometry{top=0.5cm, bottom=0.5cm, left=1cm, right=1cm}

\usepackage{fancyhdr}
\pagestyle{empty}

% space between header and content
% \setlength{\headheight}{0pt}

% indentation is zero
\setlength{\parindent}{0mm}

%----------------------------------------------------------------------------------------
%	TABLE /ARRAY DEFINITIONS
%---------------------------------------------------------------------------------------- 

% extended aligning of tabular cells
\usepackage{array}

% custom column right-align with fixed width
% use like p{size} but via x{size}
\newcolumntype{x}[1]{%
>{\raggedleft\hspace{0pt}}p{#1}}%


%----------------------------------------------------------------------------------------
%	GRAPHICS DEFINITIONS
%---------------------------------------------------------------------------------------- 

%for header image
\usepackage{graphicx}

% use this for floating figures
% \usepackage{wrapfig}
% \usepackage{float}
% \floatstyle{boxed} 
% \restylefloat{figure}

%for drawing graphics		
\usepackage{tikz}				
\usetikzlibrary{shapes, backgrounds,mindmap, trees}

%----------------------------------------------------------------------------------------
%	Color DEFINITIONS
%---------------------------------------------------------------------------------------- 
\usepackage{transparent}
\usepackage{color}

% primary color
\definecolor{maincol}{RGB}{ 0, 200, 150 }

% accent color, secondary
% \definecolor{accentcol}{RGB}{ 250, 150, 10 }

% dark color
\definecolor{darkcol}{RGB}{ 70, 70, 70 }

% light color
\definecolor{lightcol}{RGB}{245,245,245}


% Package for links, must be the last package used
\usepackage[hidelinks]{hyperref}

% returns minipage width minus two times \fboxsep
% to keep padding included in width calculations
% can also be used for other boxes / environments
\newcommand{\mpwidth}{\linewidth-\fboxsep-\fboxsep}
	


%============================================================================%
%
%	CV COMMANDS
%
%============================================================================%

%----------------------------------------------------------------------------------------
%	 CV LIST
%----------------------------------------------------------------------------------------

% renders a standard latex list but abstracts away the environment definition (begin/end)
\newcommand{\cvlist}[1] {
	\begin{itemize}{#1}\end{itemize}
}

%----------------------------------------------------------------------------------------
%	 CV TEXT
%----------------------------------------------------------------------------------------

% base class to wrap any text based stuff here. Renders like a paragraph.
% Allows complex commands to be passed, too.
% param 1: *any
\newcommand{\cvtext}[1] {
	\begin{tabular*}{1\mpwidth}{p{0.98\mpwidth}}
		\parbox{1\mpwidth}{#1}
	\end{tabular*}
}

%----------------------------------------------------------------------------------------
%	CV SECTION
%----------------------------------------------------------------------------------------

% Renders a a CV section headline with a nice underline in main color.
% param 1: section title
\newcommand{\cvsection}[1] {
	\vspace{14pt}
	\cvtext{
	\hspace{-10pt}	\textbf{\LARGE{\textcolor{darkcol}{\uppercase{#1}}}}\\[-4pt]
	\hspace{-10pt}	\textcolor{maincol}{ \rule{0.1\textwidth}{2pt} } \\
	}
}

%----------------------------------------------------------------------------------------
%	META SKILL
%----------------------------------------------------------------------------------------

% Renders a progress-bar to indicate a certain skill in percent.
% param 1: name of the skill / tech / etc.
% param 2: level (for example in years)
% param 3: percent, values range from 0 to 1
\newcommand{\cvskill}[3] {
	\begin{tabular*}{1\mpwidth}{p{0.72\mpwidth}  r}
 		\textcolor{black}{\textbf{#1}} & \textcolor{maincol}{#2}\\
	\end{tabular*}%
	
% 	\hspace{4pt}
% 	\begin{tikzpicture}[scale=1,rounded corners=2pt,very thin]
% 		\fill [lightcol] (0,0) rectangle (1\mpwidth, 0.15);
% 		\fill [maincol] (0,0) rectangle (#3\mpwidth, 0.15);
%   	\end{tikzpicture}%
}


%----------------------------------------------------------------------------------------
%	 CV EVENT
%----------------------------------------------------------------------------------------

% Renders a table and a paragraph (cvtext) wrapped in a parbox (to ensure minimum content
% is glued together when a pagebreak appears).
% Additional Information can be passed in text or list form (or other environments).
% the work you did
% param 1: time-frame i.e. Sep 14 - Jan 15 etc.
% param 2:	 event name (job position etc.)
% param 3: Customer, Employer, Industry
% param 4: Short description
% param 5: work done (optional)
% param 6: technologies include (optional)
% param 7: achievements (optional)

\newcommand{\interprise}[3]{

		\ifthenelse{\isempty{#1}}{}{

		\begin{tabular*}{}{p{0.7\mpwidth}  r}
	 		\hspace{-0.35cm}    \textcolor{black}{\Large{\textbf{#1}}} & \colorbox{maincol}{\makebox[0.3\mpwidth]{\textcolor{white}{#3}}}\\
	 	\end{tabular*}\\
	       	\cvtext{{#2}} \\

	}

}

\newcommand{\cvevent}[8] {
	
	% we wrap this part in a parbox, so title and description are not separated on a pagebreak
	% if you need more control on page breaks, remove the parbox

	\parbox{\mpwidth}{
		\begin{tabular*}{1\mpwidth}{p{0.72\mpwidth}  r}
	 		\large{\textcolor{black}{\textbf{#3}}} & {#2} \\
			\textcolor{maincol}{\textbf{#4}} & \\
		\end{tabular*}\\ \vspace{-10pt}
	
		\ifthenelse{\isempty{#5}}{}{
			\cvtext{#5}\\
		}
	}

	\ifthenelse{\isempty{#6}}{}{
		\vspace{9pt}
		{#6}
	}

	\ifthenelse{\isempty{#7}}{}{
		\vspace{9pt}
		\cvtext{\textbf{Technologies include:}}\\
		{#7}
	}

	\ifthenelse{\isempty{#8}}{}{
		\vspace{-10pt}
		\cvtext{\textbf{Réalisations:}}\\ \vspace{-20pt}
		{#8}
	}
%	\vspace{5pt}
}

%----------------------------------------------------------------------------------------
%	 CV META EVENT
%----------------------------------------------------------------------------------------

% Renders a CV event on the sidebar
% param 1: title
% param 2: subtitle (optional)
% param 3: customer, employer, etc,. (optional)
% param 4: info text (optional)
\newcommand{\cvmetaevent}[5] {
	\textcolor{maincol} {\cvtext{\textbf{\begin{flushleft}#1\end{flushleft}}}}

	\ifthenelse{\isempty{#2}}{}{
	\textcolor{darkcol} {\cvtext{\textbf{#2}}  }
	}

	\ifthenelse{\isempty{#3}}{}{
		\cvtext{{\textcolor{darkcol}{#3  \faMapMarker #4}}} \\
	}

%	\cvtext{#5}\\
}

%---------------------------------------------------------------------------------------
%	QR CODE
%----------------------------------------------------------------------------------------

% Renders a qrcode image (centered, relative to the parentwidth)
% param 1: percent width, from 0 to 1
\newcommand{\cvqrcode}[1] {
	\begin{center}
		\includegraphics[width={#1}\mpwidth]{qrcode}
	\end{center}
}


%============================================================================%
%
%
%
%	DOCUMENT CONTENT
%
%
%
%============================================================================%
\begin{document}
\columnratio{0.31}
\setlength{\columnsep}{2.2em}
\setlength{\columnseprule}{4pt}
\colseprulecolor{lightcol}
\begin{paracol}{2}
\begin{leftcolumn}
%---------------------------------------------------------------------------------------
%	META IMAGE
%----------------------------------------------------------------------------------------
\begin{center} {\includegraphics[width=4cm]{Images/pp (1).jpg}}	%trimming relative to image size
\end{center}


%---------------------------------------------------------------------------------------
%	META SKILLS
%----------------------------------------------------------------------------------------
\vspace{-2pt}
 \cvsection{José COLOMBINI}

%{\icontext{MapMarker}{12}{32, Route de la Jonelière \\44300 nantes}{black}}\\[6pt]
{\phantom{..}\textcolor{maincol}{\faGlobe}\phantom{....} Brésilien/Italien}\\[6pt]
{\icontext{MobilePhone}{12}{+33 07 63 47 34 12}{black}}\\[6pt]
{\iconemail{Envelope}{12}{jose.colombini\\ \phantom{\faEnvelope{12}} @eleves.ec-nantes.fr}{jose.colombini@eleves.ec-nantes.fr}{black}}\\
\vspace{-0.5cm}
\begin{center}
            \href{https://github.com/JoseColombini}{\textcolor{maincol}{\faGithub}}  \space\space\space \href{https://www.linkedin.com/in/jose-otavio-brochado-colombini/}{\textcolor{maincol}{\faLinkedinSquare}} 
\end{center}
    \vspace{-0.3cm}

 %\href{https://github.com/JoseColombini}{Jose Colombini}
\cvsection{Compétences}


\cvtag{Python}
\cvtag{C/C++} 
\cvtag{Linux} 
\cvtag{Matlab}
\cvtag{Simulink}
\cvtag{ROS/Gazebo}
% \cvtag{Gazebo}
\cvtag{Xtext}
\cvtag{Métamodèle}
\cvtag{MongDB}
\cvtag{VHDL}
\cvtag{SQL}
\cvtag{LabView}
\cvtag{GitHub}
\cvtag{LaTeX}
\cvtag{Assembly}
\cvtag{SiemensNX}
\cvtag{Agile}
\cvtag{OpenCV}
\cvtag{Scrum}
\cvtag{Kanban}\\
\cvtag{Direction}
\cvtag{Travaux en équipe}\\
\cvtag{Documentation}
\cvtag{Génie Logiciel}

 
\cvsection{FORMATION}
\cvmetaevent
{2021 - 2023}
{Double Diplôme d'Ingénieur - Spécialité Robotique \\ Option Professionnelle R\&D}
{École Centrale Nantes }
{France}%{Études de ingénierie généraliste (je veux faire la spécialisation en robotique) par 2 ans pour obtenir le diplôme de la ECN}
{}
\cvmetaevent{2018 - 2023}
{Ingénierie Électrique - spécialité: informatique}
{Escola Politécnica da USP } {Brésil}{}%{Études des toutes les classes de la zone électrique: contrôle, informatique, énergie, électronique et télécommunication}


\cvsection{Langues}

%\begin{center}
\cvtag{Portugais}
\cvtag{Anglais}
\cvtag{Français}
% \begin{tabular}{l l }
%  Portugais & \textcolor{maincol}{Maternelle (Brèsil)} \\ 
%  Anglais & \textcolor{maincol}{C1 (Prévu aujourd'hui)} \\ & \textcolor{maincol}{B1 (TOEFL 2016) }\\  
%  Français & \textcolor{maincol}{B1 (Prévu aujourd'hui)} \\ & \textcolor{maincol}{}    
% \end{tabular}
%\end{center}







%---------------------------------------------------------------------------------------
%	EDUCATION
%----------------------------------------------------------------------------------------


%---------------------------------------------------------------------------------------
%	CERTIFICATION
%----------------------------------------------------------------------------------------
% \newpage
% \cvsection{CERTIFICATIONS}

% \cvmetaevent
% {LPIC 1 - Linux administrator}
% {}
% {}
% {Certificate issued by the Linux Professional Institute to prove abilities in Linux administration}

% \cvmetaevent
% {IBM InfoSphere Advanced DataStage Essentials}
% {}
% {}
% {Intense course about the ETL technologies and the use of IBM DataStage.}

% \cvmetaevent
% {Jump Start Program}
% {}
% {}
% {Two months full-time training in object oriented programming in Java SE/EE, software development, testing and modern enterprise web-frameworks. Other topics were object oriented design patterns, test-driven development, SQL-databases and webservers in Java environment (Tomcat / Glasfish / JBoss / Jety)}


% \cvmetaevent
% {Online Classes}
% {}
% {}
% {It is important for me to stay up to date with the newest topics in the field of IT. In DevOps it is also important to have a general overview and a hands-on experience on them. Therefore, besides intense article studies, I also keep myself up to date with online classes.}

% \vfill
% \cvqrcode{0.7}

% \newpage
% \mbox{} % hotfix to place qrcode on the bottom when there are not other elements
% \vfill
% \cvqrcode{0.7}

\end{leftcolumn}
\begin{rightcolumn}
%---------------------------------------------------------------------------------------
%	TITLE  HEADER
%----------------------------------------------------------------------------------------
\fcolorbox{white}{darkcol}{\begin{minipage}[c][4cm][c]{1\mpwidth}
%	\begin {center}
%		\LARGE{ \textbf{ \textcolor{white}{ \uppercase{José Otávio  BROCHADO COLOMBINI} } } } \\
%		\textcolor{white}{ \rule{0.1\textwidth}{1.25pt} } \\[4pt]
%			\end {center}
			\begin {center}
\textcolor{white}{{\textcolor{white}{Élevé} \large{\textbf{Ingénieur Double Diplôme à Centrale Nantes.}} \large{\normalsize{specialisation}} \large{\textbf{Robotique} avec experience en \textbf{Informatique, électronique et systèmes embarques}}.\\ \normalsize{recherche} \large{\textbf{stage}} 6 mois %\normalsize{en} \textbf{\large{Systèmes Embarqués}}
}\\
\normalsize{Expérience en} \textbf{\large{Projet et R\&D}}}
    			\end {center}
% \vspace{-0.9cm}\textcolor{white}{
% \begin{itemize}
%   {\item Pratique en VHDL;
% \vspace{-5pt}   \item Experience en Projet et R\&D;} 
% \end{itemize}}
\end{minipage}} 
%\vspace{-40pt}

%---------------------------------------------------------------------------------------
%	PROFILE
%----------------------------------------------------------------------------------------


%---------------------------------------------------------------------------------------
%	WORK EXPERIENCE
%----------------------------------------------------------------------------------------
\vspace{-2pt}
\cvsection{Expérience}

\interprise{Accenture Labs - Digital Experience}{Groupe de recherche et développement de systèmes XR et interaction homme-machine dans le contexte de metaverse}{Fev 2022 - Aug 2022}



% \cvevent{}{}{Ingénieur de R&D}{Recherche et Développement en Téléopération de Robots}
%     {\begin{itemize}
%       %  \item Travail intitule \textbf{A Proposal for a Concrete Syntax for Use Case}.
%         \item Améliorer la représentation d'\textbf{Exigence Fonctionnelles (cas d'usage)} dans les processus \textbf{de Génie de Software}
%         \vspace{-5pt}\item\textbf{l'Ingénierie Dirigée par les Modèles} (IDM) et implémenté le \textbf{\textit{métamodèle}} a l'aide de \textbf{Xtext}
%         \vspace{-5pt}\item Projet \textit{open source} sur \href{}{GitHub \textcolor{maincol}{\faGithub}}
%         %\item J'ai créé une syntaxe concrète et j'ai implémenté le \textbf{\textit{métamodèle}} a l'aide de \textbf{Xtext};
%         %\item Il est \textbf{publié} au Symposium Brésilien sur les Systèmes d'Information (SBSI) \textcolor{maincol}{\href{https://sol.sbc.org.br/index.php/sbsi_estendido/article/view/15353}{\faFile}};
%         %\item Le outil \textbf{UCWriter} est un projet \textit{open-source} sur GitHub \href{https://github.com/JoseColombini}{\textcolor{maincol}{\faGithub}}.
%     \end{itemize}}{}{}{\begin{itemize}
%         \item \href{https://sol.sbc.org.br/index.php/sbsi_estendido/article/view/15353}{Publiés \textbf{A Proposal for a Concrete Syntax for Use Case} \textcolor{maincol}{\faFile}}
%     \end{itemize}}
\cvevent{}{}{Ingénieur R\&D}{Recherche et Développement}
    {\begin{itemize}
        \item Développement d'une plateform de teloperation de robots (missions autonomes)
\vspace{-5pt}        \item Robots: Boston Dynamics Spot, Franka Emika Panda
\vspace{-5pt}        \item L'utilisation: ROS, Moveit, GraphNav, OpenCV, PointCloud, Unity, ReactJS
    \end{itemize} \vspace{-5pt}}
    {}{}{}
    
 \interprise{{Skyrats\href{https://www.linkedin.com/company/skyrats/}{ \textcolor{maincol}{\faLinkedinSquare}}\href{https://www.linkedin.com/company/skyrats/}{ \textcolor{maincol}{\faGithub}}}}{Projet étudient: \textbf{recherche et développe} des \textbf{drones intelligentes} dans le département de Électronique de la Poli-USP}{Avr 2018 - Jul 2021}

    \cvevent{}{6 mois}{Chef de Projet de Hardware}
        {Organisation et Gestion}
        {\begin{itemize}
           % \item Créer un nouvelle groupe de travail et etre le premier chef;
            \item Implémenter et enseigner le \textbf{Scrum} pour les nouvelles membres;
        \end{itemize}}
        {}{}{}
    

    \cvevent{}{6 mois}{Assistant de Professeur de Systèmes-Embarques}{Enseignement et Documentation}
    {\begin{itemize}
        \item Réalisation de supporte pédagogique et enseigner le \textbf{ROS} et simulation avec \textbf{Gazebo};
\vspace{-5pt}        \item Créer un VM enviroment 
    \end{itemize}}{}{}{}


    \cvevent
    
    	{2 ans}
    	{Assistant d'Ingénieur de Systèmes Embarques}
    	{Recherche et Développement}
    	{\begin{itemize}
    	    \item Développer software \textbf{ROS} (mavLink, Mavros) en \textit{Python} et \textit{C/C++} et simuler dans le \textbf{Gazebo} utilisent les \textbf{méthodes agiles};
\vspace{-5pt}    	    \item Utilisent \textbf{vision par ordinateur} et autres capteurs par navigation;
\vspace{-5pt} \item Système avec deux ordinateurs: un Linux et un PX4 autopilot;
\vspace{-5pt} \item \textit{Responsable} pour le \textbf{projet} de Hardware;
    	   % \item Autogestion des travaux et mise en œuvre de le \textbf{méthode agile};
    	    %\item Responsable pour organiser et fait le pitch pour comité d'évaluation de projets de l'Endowment de la POLI pour obtenir des investissements.
    	\end{itemize}}
    	{}{}
    	{\begin{itemize}
    	  %  \item \textbf{Champion} sur la COBRUF Drones 2018 (Bresil);
    	    \item 5e sur 12 en la \textbf{International Micro Air Vehicle Conference and Competition (IMAV) 2019 - Madrid} outdoor compétition;
    	    %\item 6e sur 10 en la \textbf{IMAV 2019 - Madrid} indoor compétition.
    	\end{itemize}}
\interprise{\href{https://pcs.usp.br/en/}{PCS-EPUSP \textcolor{maincol}{\faLink}} | \href{https://fapesp.br/en}{FAPESP \textcolor{maincol}{\faLink}}}{Suite a présentation au comité scientifique obtenu d'une subvention pour développer un projet de recherche}{Oct 2019 - Oct 2020}

\cvevent{}{}{Assistant de rechercher}{Recherche et Développement}
    {\begin{itemize}
      %  \item Travail intitule \textbf{A Proposal for a Concrete Syntax for Use Case}.
        \item Améliorer la représentation d'\textbf{Exigence Fonctionnelles (cas d'usage)} dans les processus \textbf{de Génie de Software}
        \vspace{-5pt}\item\textbf{l'Ingénierie Dirigée par les Modèles} (IDM) et implémenté le \textbf{\textit{métamodèle}} a l'aide de \textbf{Xtext}
        \vspace{-5pt}\item Projet \textit{open source} sur \href{}{GitHub \textcolor{maincol}{\faGithub}}
        %\item J'ai créé une syntaxe concrète et j'ai implémenté le \textbf{\textit{métamodèle}} a l'aide de \textbf{Xtext};
        %\item Il est \textbf{publié} au Symposium Brésilien sur les Systèmes d'Information (SBSI) \textcolor{maincol}{\href{https://sol.sbc.org.br/index.php/sbsi_estendido/article/view/15353}{\faFile}};
        %\item Le outil \textbf{UCWriter} est un projet \textit{open-source} sur GitHub \href{https://github.com/JoseColombini}{\textcolor{maincol}{\faGithub}}.
    \end{itemize}}{}{}{\begin{itemize}
        \item \href{https://sol.sbc.org.br/index.php/sbsi_estendido/article/view/15353}{Publiés \textbf{A Proposal for a Concrete Syntax for Use Case} \textcolor{maincol}{\faFile}}
    \end{itemize}}
    
      
% \interprise{Ordre DeMolay}{Organisation de jeunesse \textbf{philanthropique} et philosophique.
\begin{itemize}
    \item Organiser des événements pour \textbf{collecter des dons}, comme les dîner bénéfice (plus 100 personnes);
   \vspace{-5pt} \item \textbf{Aider} les jeunes à s'améliorer avec mentorat.
\end{itemize}}{Depuis Mai 2017}

    
    % \cvevent{}{Aout 2020 - Dec 2020}{Monitoria}{Enseigner}
    %     {\begin{itemize}
    %         \item Aider le Professeur Marcelo Zuffo dans les cours;
    %         \item Enseigner \textbf{ROS}, \textbf{MavROS} et \textbf{Gazebo} aux étudiantes;
    %         \item Écrit matériel didactique pour les étudiantes e publier dans le GitHub de l'équipe \href{}{\textcolor{maincol}{\faGithub}}.  
    %     \end{itemize}}{}{}{}

 
    
   



    

%\cvevent{}{}{\Large{Autres Projets}}{}
%{Innovation pour le système de vaccination du état de São Paulo dans le cours de Génie Logiciel (projet approuvé) (JS/Mongo);\\
%Projet d'une version simplifie d'un ARM en VHDL et FPGA dans le cours;\\
%Organiser des évènements de discussion politique avec le bureau des élèves.}{}{}{}

% hotfixes to create fake-space to ensure the whole height is used

\end{rightcolumn}
\end{paracol}
\end{document}

